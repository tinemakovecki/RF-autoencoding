\documentclass[12pt,a4paper]{article}

\usepackage[slovene]{babel}
\usepackage{amsmath}

\begin{document}

\title{\LARGE{Zapiski za magistrsko}}
\maketitle

%===================================================%
% Bo uporabljeno za zapisovanje zadev za magistrsko. Zgodnji začetek pisanja, hopefully bo pomagalo tudi za predstavitev.

\section{Uvod}

\section{Osnovni pojmi}
Preden se lotimo zastavljenega problema moramo predstaviti osnovne pojme s katerimi delamo. Predstavili bomo, kaj so avtokodiriniki in naključni gozdovi ter nekatere osnovne lastnosti, npr. kako avtokodirnike testiramo. Argumentirali bomo tudi zakaj so naključni gozdovi zelo dobro orodje, s čimer želimo pokazati, da jih je smiselno uporabiti namesto nevronskih mrež.
% TODO: je treba povedati še kaj o osnovah strojnega učenja? Najbrž, najbrž

\subsection{Avtokodirniki}
% TODO:
% kaj je avtokodirnik
% zakaj je uporaben
% kako ga testiramo

\subsection{Naključni gozdovi}
% TODO:
% kaj je naključni gozd
% zakaj je uporaben: argument, da z kombiniranjem dreves v gozd dobimo boljši rezultat?
% !!! prednosti pred nevronskimi mrežami 
% !!! kaj želimo doseči s tem, da ga uporabimo: hitrost, alternativno implementacijo, ! vpogled v delovanje avtokodirnika ! (mogoče v posebej section)

\section{Cilj}  
V magistrskem delu želimo z uporabo naključnih gozdov implementirati avtokodirnik. Utemeljili bomo, da so avtokodirniki uporabno orodje, vendar v delovanje standardnih avtokodirnikov praktično nimamo vpogleda in tako tudi rezultatov pridobljenih z njimi ne moremo dobro razumeti. Z uporabo naključnega gozda želimo doseči alternativno implementacijo, katere delovanje ne bo t.i.''black box''. Z analizo delovanja avtokodirnika na nekaterih primerih bomo utemeljili, da imamo v njegovo delovanje boljši vpogled. Želimo pa tudi testirati, kako dobro bo avtokodirnik deloval v primerjavi s standardnim pristopom.

\end{document}